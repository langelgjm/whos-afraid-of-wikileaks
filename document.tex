%%This is a very basic article template.
%%There is just one section and two subsections.
\documentclass[12pt]{article}

\usepackage{graphicx}
\usepackage{endnotes}
\usepackage[round]{natbib}
\usepackage{setspace}
\usepackage[margin=1in]{geometry}
\usepackage[nolists]{endfloat} % to put all floats at end for draft submission
%\usepackage{url} % required to get natbib to correctly process internet resources

\def\citeapos#1{\citeauthor{#1}'s (\citeyear{#1})} % define a possessive citation command

%\renewcommand\makeenmark{\textsuperscript{\theenmark}}

\title{Who's Afraid of Wikileaks? \\ \vspace{2 mm} \large Missed Opportunities in Political Science Research}
%\author{Gabriel J. Michael \\
%		Department of Political Science \\
%		The George Washington University \\
%		}

\begin{document}

\maketitle

\doublespacing

\begin{abstract}
Leaked information, such as Wikileaks' Cablegate, constitutes a unique and valuable data source for researchers
interested in a wide variety of policy-oriented topics. Yet political scientists have avoided using
leaked information in their research. This article argues that we can and should use leaked information 
as a data source in scholarly research. First, I consider the
methodological, ethical, and legal challenges related to the use of leaked information in research, concluding 
that none of these present serious obstacles. 
Second, I show how political scientists can use leaked information to generate novel and unique 
insights about political phenomena using a variety of quantitative and qualitative methods. Specifically, 
I demonstrate how leaked documents reveal important details about the Trans-Pacific Partnership 
negotiations, and how leaked diplomatic cables highlight a significant disparity between the U.S. governments' 
public attitude towards traditional knowledge and its private behavior.
\end{abstract}

\section{Introduction}

In 2010, Wikileaks became a household name. A series of three high profile ``mega-leaks'' 
catapulted the organization to international notoriety.
The ``Afghan War Diary'' contained thousands of military logs relating to the U.S. war in Afghanistan;
the ``Iraq War Logs'' made public nearly 400,000 U.S. 
Army field reports from Iraq; and ``Cablegate'' resulted in the release of a quarter-million 
diplomatic cables sent from posts around the world to the U.S. State Department.
Media coverage initially focused on the substance of the leaks, which were as likely to contain 
important revelations about interstate behavior as gossip about political celebrities \citep{shane2010wikileaks,hooper2010silvio}.
Increasingly, however, interest shifted away from the content of the leaks and towards questions 
about Wikileaks as an organization, its founder Julian Assange, and the relationship of the organization 
to traditional media \citep{saunders2011wikileaks,benkler2011free}.

While political scientists have been somewhat willing to study the politics of leaking and leakers, 
they have largely avoided engaging with leaked information itself.
Most of the discussion surrounding leaks and leakers has focused on 
their consequences for democracy and diplomacy, or the role of 
leakers as a new kind of political actor \citep{simmons2011international,
davis2012political,pieterse2012leaking,springer2012leaky,
wong2013e-bandits}. An editor's note in this very journal declared ``the explosion of information 
from Wikileaks will provide innumerable opportunities for analysis about access to information, social action 
defending that access, and the political ramifications resulting from these actions,'' \citep[123]{gore2011editors}.

But published political science research that actually relies upon leaked information is surprisingly rare, especially given the breadth and depth of information that leaks have made 
available. 
In fact, \emph{International Studies Quarterly} (ISQ), a major international relations journal, has adopted a provisional policy against handling manuscripts that make use of leaked documents if such use could be construed as mishandling classified material. According to the journal's editor, this policy prohibits direct quotations as well as data mining, and was developed in consultation with legal counsel. Stating that editors are currently ``in an untenable position,'' he noted that ISQ's policy will remain in place pending broader action from the International Studies Association (ISA, publisher of ISQ and several other disciplinary journals), which is discussing how to proceed (D. Nexon, personal communication, August 8, 2014).\endnote{ISQ's policy is not shared by all of the journals published by the ISA.}
Among works published in \textit{Review of Policy Research}, only a single article mentions Wikileaks, and then only in passing \citep{kingiri2012role}.
In other venues, researchers occasionally cite a few leaked cables from 
Cablegate with little discussion \citep{bowen2011irans,guliyev2012political}.
\citet{schrodt2012precedents} notes that the Afghan War Diary and Iraq War Logs are examples of 
geolocated event data, and thus could prove useful for analysis, but does not actually analyze the 
data.
\citet{mouritzen2012explaining} use information from Cablegate to supplement their analysis of the 
Russo-Georgian war, while
\citet{petersohn2013effectiveness} examines the effectiveness of private contractors with data from the 
Iraq War Logs. 
One notable exception to the lack of engagement with leaked data in political science 
can be found in the \emph{Middle East Review of International Affairs}, which has published several 
articles relying on leaked diplomatic cables in analyses of regional politics in Turkey, Israel, 
Ethiopia, Eritrea, and Lebanon \citep{altiparmak2011wikileaks,
spyer2011israel,lefebvre2012choosing,smyth2011``independent}.

Outside of academic journals, blogs such as \textit{The Monkey Cage} have been slightly more willing 
to treat leaked information as a data source \citep{voeten2010wikileaks,michael2013united}.
But the trend at conferences has shifted towards less engagement. In 2012, the first year following the 
complete and unredacted release of Wikileaks' diplomatic cables, the annual meeting of the International Studies 
Association (ISA) hosted sixteen papers mentioning leaks, although only about half of these actually dealt 
with the content of the leaks. By contrast, in 2013, no papers dealing with leaks were presented, and in 2014, 
only two were presented, and only one of these made use of leaked information.

Scholars in other disciplines have been more willing to make use of leaked information. 
In fields as varied as informatics, applied mathematics, geography, and economics, researchers have 
enthusiastically turned to the leaked information of the Afghan War Diary and the Iraq War Logs 
as invaluable data sources for modeling and predicting conflict \citep{oloughlin2010peering,linke2012space-time,zammit-mangion2012point,cseke2013sparse,rusch2013model,
zammit-mangion2013modeling}.
Indeed, \citet{dedeo2013bootstrap} say that the Afghan War Diary ``is likely to become 
a standard set for both the analysis of human conflict and the study of empirical methods for the 
analysis of complex, multi-modal data,'' (p. 2257). Legal scholars have also begun to discover 
the value of the Cablegate corpus as a data source. \citet{khoo2011what} and \citet{mendis2012destiny} 
both cite Wikileaks cables to support their analyses of international relations in Asia. 
More recently, \citet{el_said2012morning} quotes extensively from 
leaked diplomatic cables to elucidate the bilateral free trade agreement negotiations between the United States 
and Jordan. 

It is not obvious why political scientists are more reluctant to engage with 
leaked information as a data source than scholars in other disciplines, but my disciplinary 
diagnosis suggests two broad reasons. First, some scholars have concerns about data quality, 
ethics, and the legal and professional consequences associated with using leaked data. 
Second, some scholars believe that leaked data offers no additional value for research, 
or is not particularly relevant to the study of political science.

Both of these sentiments are incorrect. 
In this article, I argue that political scientists \emph{can} use leaked information because 
the methodological, ethical, and legal objections raised to its use are overstated. 
Furthermore, I argue that political scientists \emph{should} use leaked information because it 
is uniquely valuable and offers insights that would otherwise be unavailable.

In the first part of the article, I review some of the methodological, ethical and legal objections to the use of leaked data 
in research. I conclude that 
questions of data quality and selection bias are no more pressing for leaked information 
than for more common data sources. Furthermore, most ethical concerns are only relevant for those deciding 
whether to leak information in the first place or members of the media who can provide benefits 
such as notoriety to potential leakers. However, some legal concerns merit closer attention, particularly for researchers 
who hold security clearances or those who have worked, currently work, or in the future wish to work for the U.S. government.

With respect to Wikileaks in particular, one common perception, in part influenced by media 
coverage, is that leaked data contain little more than confirmations of common knowledge or 
gossip \citep{saunders2011wikileaks,chesterman2011wikileaks}. But as \citet{bob2010wikileaks} argues, this perception is 
misguided. In the second part of the article, I use leaked information from multiple sources, including Wikileaks' 
Cablegate and recently 
released documents relating to the Trans-Pacific Partnership agreement, to provide three examples 
demonstrating the kinds of research that leaked information---and only leaked information---makes possible. 
My examples illustrate how leaked information can inform policy-relevant research in a variety of areas, such as 
innovation and intellectual property policy, trade policy, and traditional knowledge.
These examples demonstrate 
how both quantitative and qualitative research in political science can benefit from the use of 
leaked information as a data source. 

While I focus largely on U.S. policy within my own area of research, my conclusions are broadly relevant 
to political scientists, policy analysts, activists, and advocates working in a variety of areas. 
Leaked information that is currently available sheds light on the politics and policy of climate negotiations, 
international labor and environmental standards, public health, and international development. Nor are leaks 
purely a U.S.-centric phenomenon: \citet{zayani2013jazeeras} discusses the importance of Al Jazeera's 
``Palestine Papers'' for Middle Eastern media politics, and we can expect leaks from a variety of sources in the future. 
Rather than dismissing leaked information, we ought to engage with it. This article shows both why and how.

%\citep{rusch2011modeling} research paper on mortality rates

% Interestingly, among authors that explicitly analyze leaked data, most are foreign.

% Furthermore, methodological, legal and ethical concerns have not stopped scholars in 
% other disciplines from using leaked data in their research. 

\section{Methodological, Ethical, and Legal Concerns}

\subsection{Methodological Concerns}

Data quality is a key concern for scholars who might wish to use leaked information as a data source. 
Leaked information is by its very nature illicit, with the result that its provenance is often unknown. 
This can make it difficult to assess the authenticity 
or reliability of such data \citep{kelly2012wikileaks:}.

However, concerns about the data quality of leaked information are no more pressing for 
leaked information than for most formally released information. Interview and archival research are 
fraught with potential data quality pitfalls, but these challenges are not a reason to avoid 
using such research methods. As \citet{hecht2012being} explains,
\begin{quote}
Interviews do not offer an unmediated window on events. But neither do documents. Both 
types of sources are products of their times and circumstances. Both make certain events visible while 
leaving others invisible. They are generated for particular audiences, and the stakes and nuances 
of their narratives are often difficult to discern (pp. 341-342).
\end{quote}
Nor is data quality only a relevant concern for qualitative research: 
quantitative research frequently takes for granted the quality of its datasets, even 
when such datasets have well-known problems \citep{herrera2007improving}.
As with any data source, scholars should 
carefully consider the quality of the information and attempt to verify its authenticity and 
reliability when possible. With respect to leaked information in general, such verification might be 
achieved by 
comparing portions of leaked data to data from other sources, or corroborating the events or accounts related 
in leaks to newspapers or other public records.
With respect to Wikileaks in particular, it is worth noting that significant portions of its releases  
have been vetted by well-respected news organizations, such as \emph{The New York Times}, 
\emph{The Guardian}, and \emph{Der Spiegel} \citep{debevec2010professor:}. 
\emph{The Guardian} relied upon regional specialists and cross-checked material with other sources, while
the \emph{Times} had reporters familiar with secret documents verify their 
authenticity \citep{kirchner2010gaining,morisy2010bill}. Since news organizations naturally focus on 
material deemed newsworthy, researchers analyzing less newsworthy material should be sure to engage in 
their own vetting process.\endnote{Thanks to an anonymous reviewer for this point.}

Selection bias presents another potentially serious problem. Most leaks represent only a fraction 
of a much larger collection of information, and the selection process employed is  
unknown.\endnote{For example, according to State Department 
officials, during the 2006-2010 period from which most of Cablegate's quarter million cables originate, 
the State Department transmitted 2.4 million cables through other systems \citep{roberts2012wikileaks:}. 
Thus Cablegate contains only about one out of every ten cables transmitted.}
In other cases, organizations such as 
Wikileaks explicitly engage in selection and editing before release, potentially introducing bias.
But again, these problems are not unique to leaked information. Quantitative datasets with limited 
coverage frequently constrain researchers, forcing them to focus on only the countries for which data 
are available. Nor do researchers always know which data they are missing. Archives rarely contain 
complete collections of documents; rather, holdings are curated by a non-random process. While 
archivists may endeavor to retain the most important documents, importance is subjective and dependent 
upon research interests.\endnote{``The coherence of written history masks the inherent 
unruliness of research\ldots archival silences stem 
as much from choices about \emph{what} to preserve as from overt classification and 
secrecy,'' \citep[342]{hecht2012being}.}

Thus, while leaked information may not be of perfect quality, and may be incomplete or even 
biased in coverage, there is no reason to expect that leaked information is in these respects any worse 
than the commonplace data sources researchers rely upon on a daily basis. We should not necessarily 
be especially suspicious of leaked information simply because it has been leaked.
That said, it is incumbent upon researchers to recognize the limitations common to all research involving documents. 
Documents provide an incomplete picture of events, often from a one-sided perspective. 
Leaked documents and data may be out of date, or have played no role in decision making or 
policy making processes.
But these common problems can be mitigated. For example, readers of Cablegate should  
remind themselves of the ``institutional and America-centric bubble'' the writers of diplomatic cables 
inhabit \citep{kinsman2011truth}.

Here, as elsewhere, 
the best approach is to proceed carefully, always remaining conscious of concerns about data quality 
and selection bias. As \citet{western2010american} notes, leaked data such as diplomatic cables are not  
novel sources of information: diplomatic cables are standard fare for scholars of U.S. foreign policy. 
Rather, the leaks simply allow us to ``leapfrog the traditional 30 years process'' of declassification.

\subsection{Ethical Concerns}

There are important ethical considerations linked to the use of leaked information in research. 
While large scale, aggregate use of data from leaks is unlikely to cause harm to individuals, 
publication of detailed content can and has 
caused harm to individuals. For example, the Taliban have admitted to studying Wikileaks' 
Afghan War Diary in an attempt to find information about Afghan civilians who provided information 
to the U.S. military \citep{winnett2010wikileaks}. Apart from potential physical harm, 
leaked information 
has also forced diplomats to step down from their posts: in the wake of Cablegate, the 
U.S. ambassador to Ecuador was expelled, and the U.S. ambassador to Mexico 
resigned \citep{romero2011ecuador,associated_press2011us}. 
Looking beyond individual harm, the major concern with most leaked information is its potential to 
cause harm to organizations and agencies, governments, or more abstract subjects such as 
national security or international reputation.
% But what about harm to organizations / nations / governments / agencies?

While sobering, these ethical issues are most relevant to the individuals or organizations deciding 
whether to publish leaked 
information that is not already publicly available. These considerations are generally \emph{not} 
relevant 
to the question of whether to use leaked but publicly available information in scholarly research. 
The Afghan War Diary, Iraq 
War Logs, and Cablegate leaks have now been publicly available to the entire world for three to four 
years. The information is widely mirrored across the Internet, stored on servers in multiple countries, 
and there is no chance of it being 
recovered and made secret again. In fact, any attempt to reign in distribution of the information at this 
late stage is likely only to increase its availability due to the ``Streisand effect'' \citep{marton2010protecting}. 
Furthermore, any retaliatory or harmful actions stemming from the leaks are likely 
to have already occurred, and in any case cannot obviously be prevented by avoiding their use in scholarly 
research.

The claim that Wikileaks has seriously harmed national security 
is difficult to support: within a few months of Cablegate, internal U.S. government 
reviews found the harm caused by Wikileaks to be relatively limited, and described the leaks as more embarrassing than 
damaging \citep{hosenball2011u.s.}. In any case, purported damage to national security or international reputation caused by leaks is likely to come soon after their initial publication (if at all), 
and not as a result of scholarly analyses published months or years after the documents themselves. 
In fact, scholarly research relying upon leaked information can offer valuable insights for the very same 
government that suffered the leaks in the first place. For example, \citet{linke2012space-time} find evidence of 
``tit for tat'' interactions between Coalition and insurgent forces in Iraq, and \citet{zammit-mangion2012point} 
develop predictive models of Afghan insurgent activity based solely on prior-year data. In these cases, 
scholarly research relying upon leaked information actually advances U.S. national security.

With respect to formal research ethics, compliance with the guidelines established by institutional review boards (IRBs) will vary by institution. This may pose challenges for collaboration between researchers at different institutions.
However, the federal law upon which most IRBs base their guidelines explicitly exempts 
``Research involving the collection or study of existing data, documents, records, pathological specimens, or diagnostic
specimens, if these sources are publicly available,'' \citep[45 CFR 46.101 (b)(4)]{u.s._government2009code}.
Leaked data that have remained widely available for several years, such at the data used in this article, 
easily fall within this exemption.\endnote{By contrast, if researchers are direct recipients of leaked information 
that is not already publicly available, the situation is very different. However, this is \emph{not} the situation 
in which the vast majority of academic researchers who might use to wish leaked data will find themselves.}
Researchers might consider taking steps to avoid drawing unnecessary attention 
to individuals identified in leaked data, for example by redacting their names. However, such tactics 
are of limited value when leaked data are publicly available, since even a simple citation or Internet search will enable 
interested readers to immediately identify the redacted information.
Fears that published research findings will cause harm by drawing attention to leaked data are unjustified, 
given that the data are already publicly available and have been widely reported on in general news media. 
Similar logic about public availability has led researchers studying data from Twitter to largely eschew 
IRB processes \citep{zimmer2014topology}.

A more pertinent ethical issue is whether researchers should make use of data that were arguably 
immorally obtained. By using leaked information, are researchers condoning the means by which it was 
obtained, or encouraging future leakers to disclose sensitive information, as \citet{conway2013why} suggests?

With respect to the first question, consider the fact that computer security researchers frequently 
analyze stolen password databases in order to 
better understand how real people choose passwords, with the ultimate goal of improving computer 
security \citep{imperva2010imperva,bbc2013`123456}.
These researchers are not condoning the theft of information; rather, 
they choose to use such information precisely because its nature makes it particularly unique and 
valuable. 

While the medical profession faces a significantly different set of concerns, it has long 
grappled with the question of how to handle ethically tainted data. According to the American Medical Association's 
Code of Medical Ethics, if ethically tainted data are the only data available, and the data are scientifically 
valid, doctors and researchers may in some cases be able to justify their 
use \citep{american_medical_association1998opinion}. Scholars who wish to use leaked information may benefit from adopting a similar approach, asking themselves whether leaked information is the only source of data, as well as whether their research will provide benefits that can justify the reliance on such data. 

In the immediate wake of Cablegate, several universities advised students to avoid commenting on, linking to, 
or even reading leaked documents, 
fearing that by doing so students might jeopardize their employment opportunities with the U.S. 
government \citep{grinberg2010will}. However, the U.S. State Department, to which the recommendation 
was originally attributed, soon clarified that it had no formal policy on the matter, and at least one of the 
universities later formally retracted its recommendation \citep{gustin2010columbia}.
In teaching, instructors need not completely refrain from discussing leaked information. Rather, instructors 
might take the opportunity to distinguish between unclassified and classified leaks, explain to 
students the potential ramifications of publicly associating themselves with classified leaks, and 
ultimately allow students to decide for themselves whether they wish to read leaked information, assigning alternative 
readings as necessary.

Researchers who have signed classified information nondisclosure 
agreements should 
carefully consider whether accessing or using classified leaked information violates those agreements, 
but researchers who have not signed such agreements do not face the same potential legal 
obligations \citep{cox2011wikileaks}. The latter group should carefully consider 
how publishing research using leaked information might affect their future prospects for access to 
classified information through formal means, as well as potential future employment that might require 
a security clearance.
Likewise, researchers may encounter significant obstacles in obtaining government grants if their research 
relies upon information leaked from the same government.\endnote{Thanks to an anonymous reviewer for this point.} 
However, researchers who do not 
have other legal obligations preventing them from using leaked information should be free to decide for themselves 
whether they believe the benefits of their research outweigh the challenges and potential negative effects it might have.

With respect to the second question, an analogy can be made to the ``fruit of the poisonous tree'' 
doctrine in law, which forbids the use of subsequent information derived from illegally obtained 
evidence. This principle exists largely to discourage police from 
gathering evidence through illegal means: if illegally obtained evidence were permitted in one 
case, this would provide an incentive for police to illegally obtain evidence in future cases. 
However, with respect to leaked information, it is not clear 
that scholarly use of such information will incentivize future leaks.
While media coverage may play a role in encouraging future leaks by providing 
potential leakers with the possibility of notoriety, the same cannot be said of scholarly research, 
which generally receives little publicity and rarely covers newsworthy issues in a timely fashion.
Furthermore, it appears that high-profile leakers such as Chelsea Manning, Edward Snowden, and 
Thomas Drake have 
been motivated to make their disclosures because of dissatisfaction with government policies, as 
opposed to a desire for publicity or notoriety. Indeed, Wikileaks' dedication to protecting the anonymity of 
its sources suggests that many leakers have no interest in publicity.

Several commentators have suggested that these leaks may harm future scholarship by causing 
diplomats, military personnel, and bureaucrats to avoid documenting information in permanent 
formats \citep{drezner2010why,simmons2011international}. 
But the claim that in response to leaks, government 
officials will, e.g., prefer the telephone to e-mail misses the fact that for the vast majority of 
content in leaks, these modes of communication are not interchangeable due to fundamental factors such as the 
amount of information that must be communicated, time differences between offices, and the availability of 
personnel. Furthermore, even if it proves true, choosing not to study already leaked 
information is unlikely to affect 
such decisions, which will be (and likely already have been) made in response to 
the leaks themselves, not their subsequent use by scholars.

\subsection{Legal Concerns}

As noted above, researchers who have signed or intend to sign agreements or otherwise gain access to classified information 
may face special considerations when deciding whether to access leaked information. Likewise, federal government 
employees may also not be permitted to 
access classified information without clearance even if the information is publicly available. However, 
leaked information is not the same as classified information. In fact, with respect to 
Wikileaks' release of diplomatic cables, more than half the cables are explicitly unclassified.

When asked to review prohibitions on 
the publication of classified defense information, the Congressional Research Service 
declared, ``CRS is aware of no case in which a publisher of information obtained through unauthorized disclosure by a government
employee has been prosecuted for publishing it,'' \citep[16]{elsea2013criminal}.
The same report emphasizes that while leaked information related to national defense may be covered by the Espionage Act, 
``There appears to be no statute that generally proscribes the acquisition or publication of diplomatic cables, although
government employees who disclose such information without proper authority may be subject to prosecution,'' \citep[14]{elsea2013criminal}.
Even for information related to national security, it would be very difficult to argue that academic research involving information 
that has been publicly available for years is somehow likely to cause harm to national security.

In general, the potential negative effects of using leaked information must be weighed against the value that such 
information may hold for scholarly research. In the following section, I argue that scholars should 
seriously 
consider using leaked information as an additional data source in the course of their research. 
When given careful attention, leaked documents can provide insights into international politics that 
would otherwise be impossible to obtain. Furthermore, because of the nature and quantity of documents 
that have been leaked, scholars can use both qualitative and quantitative techniques to gain such 
insights.

\section{The Unique Value of Leaked Information}

Historically, leaked information has been fodder first for journalists, and later for historians. 
Some leaks, such as Daniel Ellsberg's Pentagon Papers, have proved useful for research in 
presidential power, bureaucracy, and civil-military relations \citep{schwab2006clash,stevenson2006warriors}. However, many politically salient 
leaks, such as Watergate or the Plame affair offer little information directly relevant to scholarly research. 
These kinds of leaks have contributed to the perception that little of value can be gained by studying leaked information. 

New leaks are different, both in kind and in amount. In this section I offer three examples showcasing the kinds of research that 
leaked information---and only leaked information---makes possible. These three examples each use different content and different 
analytical techniques. The first example relies upon a leaked draft of the intellectual property 
chapter of the Trans-Pacific 
Partnership (TPP), a plurilateral economic treaty currently being negotiated among twelve Pacific Rim 
countries. I show how researchers can extract semi-structured information from text, and use such data 
to uncover relationships between negotiating parties.\endnote{``Semi-structured'' describes data with a 
variety of attributes but no clear or consistent schema. In contrast, structured data is represented in 
a strict format, while unstructured data provides little or no information about 
content \citep[416-418]{elmasri2010fundamentals}. Familiar social scientific quantitative methods such as regression analysis rely on structured data, while newer analytical methods such text mining begin with unstructured data before processing it into structured data.}

The second example also relates to the TPP, but relies upon a different leaked document which reports  
the remaining areas of disagreement between negotiating parties as of November 2013. This document  
covers a wide variety of substantive issues, as opposed to the previous example, which focused solely 
on a single chapter of the TPP. I show how some leaks contain structured information which 
permits the use of standard quantitative techniques to map agreement and disagreement between negotiating 
parties, as well as relative levels of contention over various substantive negotiating topics. 

The third and final example uses leaked diplomatic cables from Cablegate to show how traditional
qualitative research reveals discrepancies between the 
public and private positions taken by the U.S. government on various policy topics. Specifically, I consider 
the example of U.S. foreign policy with respect to traditional knowledge, a relatively new area of intellectual 
property primarily trumpeted by developing countries. For researchers interested in specialized topics 
that have traditionally been considered matters of ``low politics'', such as science and innovation policy, 
Cablegate offers a wealth of information that would be difficult to match even by conducting hundreds of 
interviews and visiting multiple archives, and all this without waiting decades for declassification or formal release.

All three of these examples also serve to demonstrate how leaked data offer unique insights into 
political phenomena. These or comparable data would otherwise be impossible to obtain, or in the best case scenario, 
require years or even decades of delay. The first two examples concern ongoing trade negotiations, 
which are traditionally opaque. In the case of the TPP, the United States Trade Representative (USTR) has denied 
Freedom of Information Act (FOIA) requests seeking draft texts and other information about the treaty, citing exemptions 
for national security information \citep{levine2012bring}. FOIA also poses other challenges. FOIA requests 
fare best when they clearly identify specific desired documents, but researchers often do not know what documents exist. 
Agencies have wide latitude to deny FOIA requests under the deliberative process exemption, leaving scholars 
interested in agency decision-making out of luck.
USTR is also exempt from the Administrative Procedure Act, thereby minimizing 
the agency's obligations to interact with the public \citep{kaminski2014capture}.
Even after negotiations are complete (or fail), researchers rarely have 
access to marked-up draft texts. Instead, they typically rely on interviews, reports and recollections of 
negotiators.\endnote{See, e.g., \citet{robert2000negotiating,deere2009implementation} and \citet{haggart2014copyfight:}}
Such sources are necessarily incomplete, come with embedded national bias, and do not lend themselves 
to quantitative analysis. The third example concerns diplomatic posturing on a specific aspect of 
intellectual property policy (traditional knowledge). While U.S. State Department diplomatic cables are generally eventually released to the public, the process often takes several decades; for example, in early 2014, the National Archives and Records Administration 
released about 300,000 diplomatic cables dating from 1977 \citep{national_security_archive2014u.s.}. 
Researchers interested in more recent information, 
such as U.S. foreign economic policy in the post-WTO era, will find that the Cablegate leak provides a wealth of 
information that will likely not become available through official channels until 2030 or later, if ever.

\subsection{The Trans-Pacific Partnership Intellectual Property Chapter}
\label{tpp_ip}

% Structured information refers to information stored in a well-defined, easily recoverable format that 
% is ideal for computer processing. Unstructured information usually refers to free text or documents 
% whose content is unorganized, making it difficult to process by computer. In between these two extremes 
% lies semi-structured information: bits of structured information within unstructured information, 
% or organized information that can be extracted from unstructured information without resorting to 
% full-fledged text mining.
% This and other sections need to be rewritten to meet Reviewer 3's objections. 
% Cite the Fundamentals of Database Systems, Chapter 12.1, page 416 for the trichotomy
% Can also cite the ``Is Big Data creepy'' paper

Leaks of draft treaties sometimes contain information about the positions 
of negotiating parties. This was the case in 2010 when an internal European Union document was 
leaked, revealing detailed positions of the parties negotiating the Anti-Counterfeiting 
Trade Agreement (ACTA), a plurilateral treaty focusing on the enforcement of intellectual property 
rights. Similarly, in 2013, Wikileaks released the draft text of TPP's intellectual property chapter, 
including information about country positions on specific provisions in the draft.

The bracketed text and negotiating positions in such leaks are often substantively interesting on 
their own. For example, the leaked TPP chapter contains a statement noting that the treaty should 
not prevent countries from addressing public health crises, even if doing so requires derogating from 
the treaty's obligations with respect to patented medicines. That is, the treaty permits states to 
prioritize public health over patent enforcement. Conflict between public health and patents has been 
a major concern of developing countries since the late 1990s, when patents on anti-retroviral medications 
made their prices prohibitive for the countries most afflicted by HIV/AIDS.
In the treaty, public health crises are defined to 
explicitly include the ``big three'' diseases of HIV/AIDS, tuberculosis, and malaria. 
However, the United States opposes the inclusion of a specific reference to Chagas disease \citep[Art. QQ.A.5(a)]{wikileaks2013wikileaks}. 
Chagas disease is one of several neglected tropical diseases that, while afflicting millions, 
generally receives far less funding and attention than the ``big three.'' U.S. opposition to 
explicitly mentioning Chagas disease derives from a desire to limit as much as possible 
the potential cases in which other countries can override patents in the interests of public 
health \citep[11]{government_accountability_office2007intellectual}.

Apart from the details, such leaks may contain semi-structured data. 
Negotiating positions are often indicated by standard country codes, and when multiple parties 
support or oppose a provision, their codes are listed together. These codes are functionally 
equivalent to the tags of a markup language, and 
can be programatically extracted using basic text processing tools. 
The extracted information can then be parsed to reveal the frequency with which pairs of countries (i.e., dyads) appear together on the same side of an issue, 
or the proportion of a country's proposals not supported by any other negotiating party, thereby 
providing a detailed, behind-the-scenes snapshot of the negotiating 
process.\endnote{For methodological and technical details, see the Appendix, which also includes information about 
how to access data and code for replication purposes. At first glance, this draft text might appear to be ripe for text analytics. For example, by associating bracketed text with country codes, one could potentially compare the actual content of parties' negotiating positions, rather than only the relationships between parties. In practice, 
this proves difficult. While parties are consistently identified, the draft text does not denote support or opposition for provisions in a consistent manner. Given the relatively short length of the text and the legal content of the document, many found it easier to simply manually provide written comments and 
annotations \citep{geist2013tpp,kaminski2013tpp}.}

The TPP currently comprises twelve negotiating parties: the United States, Japan, Canada, Mexico, 
Australia, New Zealand, Peru, Chile, Singapore, Malaysia, Vietnam, and Brunei. Together, 
these parties form 66 unique, unordered dyads. 

The frequencies with which these dyads appear in the text offer insight into which countries' interests overlap, 
and which groups of countries are 
aligning with one another. Intuitively, the more frequently two countries appear together on the same 
side of an issue, the more likely it is that their interests coincide.\endnote{Some combinations 
of parties might instead result from strategic choices, in which a country supports provisions it 
considers unimportant or even undesirable in exchange for support from other countries on more important provisions. 
But such strategic maneuvering could not produce aggregates that are at odds with a country's interests. 
Even a cursory reading of the draft text reveals significant coincidence 
between parties' expressed positions and the positioning we might expect based on the foreign policy of the 
parties.} Relative frequencies of 
dyads may indicate the degree to which a country's positions are acceptable to all other parties; if relatively 
few parties are willing to join in supporting a particular country, this suggests the latter country's 
positions may represent an outlier in the negotiations. Alternatively, lower frequencies of positions, 
and thus lower relative frequencies of dyads, might instead represent a degree of satisfaction with 
the current draft text. These two interpretations are not mutually exclusive: countries that are 
satisfied with the current draft text may also be outliers in the negotiations.

To represent the relative frequencies of these dyads in an easily understandable form, 
Figure \ref{fig_tpp_network_graph} presents a network graph. Network graphs and network analysis can help reveal relationships and mechanisms of information exchange between states \citep{kinne2013network}.
Figure \ref{fig_tpp_network_graph}'s connections are weighted and colored 
according to the relative frequency of dyads. That is, the thicker and darker a connection between 
two countries, the more frequently those countries appear on the same side of an issue in the draft 
text of the TPP's intellectual property chapter. I also include symmetric dyads (i.e., a country paired with itself) as a loop linking the country back to itself. The thicker and darker this loop, the more frequently a country appears in 
the text without any other negotiating parties joining it.

\begin{figure}
\caption{Country Dyad Frequencies in the TPP IP Chapter}
\label{fig_tpp_network_graph}
\centering
\includegraphics[width=1.1\textwidth]{tpp_network_all}
\end{figure}

Figure \ref{fig_tpp_network_graph} serves to highlight several notable patterns. First, the 
United States and Japan are relatively 
weakly linked to almost all other negotiating parties. The United States' strongest link is with 
Australia, a staunch U.S. political and economic partner. Canada, which formally joined the TPP 
negotiations in late 2012, has the highest number of proposals not joined by any other party, followed 
by the United States and Japan. However, when considering these proposals as a proportion of each 
country's total number of appearances in the text, Japan comes first, followed by the United States 
and Canada. Very strong links tie Singapore, Malaysia, Chile, and New Zealand together.

Given that the network graph is based solely on the TPP's intellectual property chapter, 
its basic structure makes intuitive sense given the public positions of some of the TPP parties. 
For example, the United States has a long history of pursuing strict intellectual property provisions 
in trade agreements, while middle-income countries such as Vietnam and Peru have tended to adopt more 
flexible interpretations. Thus, we would expect more connections between high-income countries and between 
middle-income countries, but few links between the two groups. However, the weakness of particular 
connections is surprising. The United States-Canada dyad, 
for example, ranks 54th out of the 66 total dyads in frequency of appearance. Furthermore, because Canada has a 
number of strong ties to other parties, the weak United States-Canada link cannot be attributed to 
both parties being outliers or equally satisfied with the draft text, as one might argue for the 
United States-Japan dyad. This suggests that Canada continues to resist the United States' intellectual property 
agenda, not just domestically, but also in plurilateral negotiations. \citeapos{michael_geist2013trans} legal analysis of Canada's positions lends additional 
support to this interpretation; he describes Canada as ``pushing back against many U.S. demands.''

\subsection{Trans-Pacific Partnership Negotiating Positions}
\label{tpp_np}

About one month after releasing the draft text of the TPP, Wikileaks released two 
additional documents related to the TPP. The first was a collection of extracts from an ``internal 
government commentary on the state of the TPP negotiations,'' and the second was a list of each 
country's negotiating positions on 87 different proposals, divided into fourteen topics \citep{wikileaks2013second}.

While the first document received significant 
attention within activist circles and by specialist news sources, given that Wikileaks 
explicitly says that it had selected the extracts from a larger document, it is impossible to know if the  
extracts are representative of the complete document. In contrast, the second document, although 
not as provocative at first glance, proves to be much more useful for 
analytical purposes. This document was prepared by an unspecified 
negotiating party to assist its delegation in identifying the proposals where disagreement remained. 
Thus, the document contains only proposals for which at least one party was preventing unanimity at 
the time of document's preparation in November 2013.

For every proposal, the document reports whether each country accepts the proposal, rejects the 
proposal, 
or has a ``reserved position.''	Given that the data are already conveniently structured as a matrix, 
the only processing required is to code these three values. The categories 
readily lend themselves to an ordinal coding scale. I code rejection as $-1$ and acceptance as 
$1$. A ``reserved position'' could reasonably be assumed to fall somewhere between rejection and acceptance, 
but given that reserved positions account for only 112 of the total 1044 values, I treat them as 
missing data to avoid making unnecessary assumptions. In addition to the 112 reserved positions, 
there are 11 country-proposal combinations with no reported data in the leaked document.

After coding, any number of standard techniques can be applied 
to analyze the leaked data. For example, by applying a distance function to the columns or rows of 
the matrix, one can obtain a matrix specifying the dissimilarity between negotiating 
parties or between proposals. 

Table \ref{tbl_tpp_dist} uses the common Euclidean distance function 
to produce a distance matrix for the TPP's twelve negotiating parties, comparing their positions across all 
87 proposals.\endnote{Other distance functions such as Manhattan distance or cosine similarity 
could also be used. For a helpful overview of common distance functions, see 
Table 9.4 in \citep[487]{sullivan2012introduction}.}

% latex table generated in R 2.15.2 by xtable 1.7-1 package
% Fri Nov  7 10:51:47 2014
\begin{table}[ht]
\centering
\caption{Distances between Negotiating Positions of TPP Countries}
\label{tbl_tpp_dist}
\begin{tabular}{lrrrrrrrrrrrr}
  \hline
 & AU & BN & CA & CL & JP & MX & MY & NZ & PE & SG & US \\ 
  \hline
AU &  &  &  &  &  &  &  &  &  &  &  &  \\ 
  BN & 11.0 &  &  &  &  &  &  &  &  &  &  &  \\ 
  CA & 9.7 & 9.3 &  &  &  &  &  &  &  &  &  &  \\ 
  CL & 12.7 & 8.1 & 10.7 &  &  &  &  &  &  &  &  &  \\ 
  JP & 9.8 & 8.5 & 9.2 & 9.8 &  &  &  &  &  &  &  &  \\ 
  MX & 10.4 & 9.0 & 7.7 & 10.0 & 9.6 &  &  &  &  &  &  &  \\ 
  MY & 11.6 & 7.9 & 10.1 & 9.1 & 9.6 & 10.4 &  &  &  &  &  &  \\ 
  NZ & 9.4 & 7.7 & 9.7 & 9.1 & 8.6 & 10.4 & 9.9 &  &  &  &  &  \\ 
  PE & 12.5 & 8.7 & 11.0 & 8.4 & 10.1 & 9.8 & 10.3 & 10.0 &  &  &  &  \\ 
  SG & 10.3 & 6.9 & 10.2 & 9.3 & 7.7 & 10.0 & 9.3 & 7.4 & 9.3 &  &  &  \\ 
  US & 11.6 & 14.4 & 13.2 & 15.7 & 13.1 & 13.8 & 16.0 & 14.6 & 15.0 & 13.9 &  &  \\ 
  VN & 12.0 & 8.2 & 10.0 & 9.4 & 9.6 & 9.8 & 7.2 & 9.5 & 9.9 & 9.3 & 15.8 &  \\ 
\hline
%\scriptsize{*Vietnam is deliberately omitted from the columns, since the lower triangle of the matrix is 
%sufficient to provide distances between all parties.} \\
%   \hline
\end{tabular}
%\raggedright
\end{table}

While the previous section relied on semi-structured data 
drawn from the draft text of a single chapter of the TPP, this section relies on structured information 
relevant to the entire treaty. The results paint a somewhat different picture. In the intellectual 
property chapter, both the United States and Japan had relatively few links to other negotiating 
parties. Here, the United States is uniquely positioned, as is evident from the large distances 
between it and every other negotiating party in Table \ref{tbl_tpp_dist}. 
In fact, the distances that exist between the United States' overall negotiating position 
and that of all other parties are large enough to consistently place the United States by itself when 
using cluster analysis, irrespective of the number of clusters chosen.

As noted above, one can also create a matrix to measure distances between 
individual proposals, or by averaging or summing groups of proposals, distances 
between the various chapters of the TPP. Recall that in this context, ``distance'' refers to 
degrees of agreement or disagreement. When analyzing distances between negotiating parties, 
it makes sense to consider the distances between country dyads, as in Table \ref{tbl_tpp_dist}. 
However, this approach makes less sense when applied to distances 
between chapters. Rather than distances between dyads of chapters, it would be more interesting to see 
the relative distance between all chapters.

Multidimensional scaling (MDS) offers a suitable solution to this problem. MDS reduces the distances 
between objects in higher-dimensional spaces to more easily understood lower-dimensional (i.e., two or three dimensional) 
spaces. Applying MDS to a matrix of distances representing the relative disagreement between TPP chapters results in a set of 
two-dimensional 
points which can be easily plotted, as shown in Figure \ref{fig_tpp_chapters}.\endnote{I used the 
\emph{cmdscale} command in R, which performs metric multidimensional scaling. The goodness-of-fit 
statistic is $> 0.88$, indicating that the two-dimensional plot captures a large amount of the 
original information. For additional details, refer to the Appendix.} 
In this figure, the farther away a point lies from the origin, the more disagreement 
exists among negotiating parties over the issues in that chapter.
The axes represent the $x$ and $y$ axes of a 
coordinate plane, and have no special meaning. The units are derived from the coding scheme; thus, 
absolute nominal distances are not as important as the relative distances between points.

\begin{figure}
\caption{Relative Disagreement Among Negotiating Parties in TPP Chapters}
\label{fig_tpp_chapters}
\centering
\includegraphics[width=\textwidth]{tpp_proposals_all_col_sums_preferred}
\end{figure}

The most striking feature of Figure \ref{fig_tpp_chapters} is the degree to which the TPP's 
intellectual property chapter (G) 
is separated from all other chapters. A simple count of proposals shows that the intellectual property 
chapter has more unresolved issues than any other chapter in the leaked data. However, MDS 
also reveals that even when compared to other chapters with many unresolved 
issues such as the environment chapter (D) or the market access chapter (J), the intellectual property chapter 
is relatively more contentious to a large degree.\endnote{In order to combine the data on various 
proposals within a chapter into a single row representing 
the overall degree of contention within a chapter, which can then be compared to other chapters, 
I chose to use column sums. An alternative approach would have been to use column means. 
Column sums are 
non-normalized; i.e., chapters with a larger overall number of proposals will have a wider range of 
potential distances than chapters with a smaller number of proposals. If the leaked data represented a 
random sample of proposals taken from the chapter, then using the non-normalized column sums would 
exaggerate the differences between chapters. However, because the leaked data represent 
only the remaining areas of disagreement, chapters with fewer proposals can be assumed to be less 
contentious than chapters with greater numbers of proposals. Using column means would assume that 
all chapters are potentially equally contentious, which appears not to be the case. Thus, column 
sums are a more appropriate measure of contention between chapters.}

Analysis of the leaked text reveals that this contention results largely from demands for stronger intellectual 
property protection attributed to the United States, 
coupled with resistance to these demands from most other negotiating partners. While U.S. demands for 
strong intellectual property protection are hardly surprising, U.S. isolation in voicing such demands 
is surprising---one might expect Japan, another economic powerhouse and the second largest net exporter of 
intellectual property in the world after the United States, to also voice demands for strong protection. Yet when 
considering the negotiating distances between countries in the intellectual property chapter, Japan 
is not distinguishable from a cluster of nine other countries. U.S. isolation in its demands coupled with 
its very public insistence on promoting strong intellectual property rights in the TPP indicates that the 
United States is expending significant bargaining capital on the issue of intellectual property as opposed to 
other topics, such as environmental or labor standards. Since bargaining capital is a scarce resource, this 
means the United States is prioritizing the protection of intellectual property over many other competing priorities, 
despite insistence from the USTR that it is equally committed to 
achieving its goals in other policy areas \citep{u.s._trade_representative????trans-pacific}. 

\subsection{U.S. Foreign Policy on Traditional Knowledge}
\label{tk}

``Cablegate'' refers to Wikileaks' most well-known release of information: over a quarter million 
United States diplomatic cables, primarily written between 2002 and 2010. These diplomatic cables 
contain communications between hundreds of U.S. embassies, consulates, and the Department of State. 
Initially, Wikileaks released small numbers of cables, and began working with traditional news 
organizations to redact and publish larger amounts. However, in late 2011, the unredacted 
contents of all cables were revealed when a reporter unintentionally published the passphrase used 
to decrypt the file containing the cables. The unredacted cables are now searchable 
using several public web-based interfaces, and available for download as a database. 
The leaked cables contain both structured and unstructured information. 
This section focuses on the use of the unstructured information contained in leaks as an alternative 
data source for traditional qualitative research.

More than half of Cablegate's cables are unclassified, about 100,000 are labeled ``confidential,'' 
and a small 
fraction are classified ``secret.'' None are classified as ``top secret.'' News reporting on the leak 
focused almost exclusively on issues perceived to be of high political importance, as well as 
the humorous and sometimes tawdry assessments of foreign officials provided by embassy and consulate 
staff. However, as one might expect given the relatively small fraction of cables classified ``secret'' 
and 
the well-documented tendency towards overclassification, much of the material 
contained in the leaked cables reports the relatively quotidian details of embassy and consulate 
work.\endnote{Overclassification results both from the lack of immediate costs for classification, 
and the psychological tendency of individuals to view secret information as being of a higher quality 
than non-secret information, thus providing an additional incentive to classify documents 
\citep{kaiser1986impact,overman1990information,aftergood2009reducing,travers2013secrecy}.} 
While this type of material is unlikely to make headlines, it represents a potential gold mine 
of information for political scientists. In their cables, foreign service officers provide detailed 
and candid assessments of their interactions with foreign bureaucrats, analysis of domestic politics, 
and reporting on local, national, regional, and international meetings. Although their accounts are 
subjective and written within professional constraints, leaked cables often contain information 
not found anywhere else.

As an example, consider the value of leaked cables in revealing the United States' 
position on the topic of traditional knowledge. As part of a current research project that asks why 
very differently situated states 
adopt remarkably similar intellectual property policies, I am examining the diffusion of traditional 
knowledge protection in the Global South. According to the World Intellectual Property Organization 
(WIPO), traditional knowledge refers to the ``knowledge, know how, 
skills, innovations or practices, that are passed between generations in a traditional context and 
that form part of the traditional lifestyle of indigenous and local communities who act as their 
guardian or custodian,'' \citep{world_intellectual_property_organization2013traditional}.

Traditional knowledge is a narrow topic within the field of intellectual property, which is itself 
only one aspect of U.S. economic policy. Nevertheless, the leaked cables contain dozens of references 
to traditional knowledge, originating from diplomatic posts all over the world. Despite being written 
by different authors in different locations, sometimes years apart, the cables' references are nearly 
always either purely descriptive and neutral, or analytical and negative, revealing a U.S. government 
policy towards traditional knowledge that considers it at best as a political annoyance, and at worst as 
an attempt to subvert what the United States views as mainstream intellectual property rights. Furthermore, the U.S. 
approach to traditional knowledge as detailed in the leaked cables differs significantly from the 
diplomatic statements of the United States in public fora such as WIPO.

One can get a better sense of what constitutes traditional knowledge by way of example. Turmeric is 
a South and Southeast Asian plant whose yellow roots are ground to produce a spice used extensively 
in cooking. Apart from its culinary use, turmeric is also widely used in Chinese and Indian 
traditional medicine, as well as in religious rituals and marriage ceremonies. Turmeric has 
anti-inflammatory and wound healing properties that have been known for centuries, and which in part 
account for its value in traditional medicine. The awareness, use, and documentation of these medical 
properties by Indian and Chinese people is an example of traditional knowledge.

Over the past twenty years, a significant number of African, Asian, and Latin American countries have 
adopted legislation protecting traditional knowledge as a form of intellectual property. In some cases, 
such legislation has come in response to accusations of misappropriation or ``biopiracy'' by Western 
firms, particularly of traditional knowledge associated with genetic 
resources \citep{dagne2012protection}. For example, in 1995 two researchers at the University of Mississippi 
were granted a patent on the topical and oral use of turmeric as a wound healing agent, despite such 
use being a well-known Indian folk remedy. Ultimately, the Indian government successfully challenged the 
U.S. patent, resulting in its cancellation two years later \citep{kumar1997india}.

Meanwhile, since 2001 WIPO has hosted over two dozen meetings of an intergovernmental committee 
convened to study traditional knowledge within the framework of intellectual property. Political 
divisions within this committee continue to stymie progress. In general, developed countries 
oppose the creation of a binding international instrument protecting traditional knowledge, preferring 
instead to limit such protection to the national level. On the other side of the issue, many 
developing countries are pressing for a binding international treaty similar to the international 
intellectual property treaties protecting copyrights or patents, or the possibility of an amendment 
to the World Trade Organization's Agreement on Trade Related Aspects of Intellectual Property Rights 
(TRIPS) that would require disclosure of the sources of any traditional knowledge relied upon 
in patent applications. Interestingly, this debate flips the typical politics 
of intellectual property on its head: with respect to traditional knowledge, developing countries are 
calling for stricter rules, while developed countries emphasize the free flow of information and the 
importance of the public domain.

For its part, the United States does not officially reject the possibility of a binding international 
instrument on traditional knowledge, instead claiming that it supports the committee's work ``without 
prejudice to the type of instrument or instruments that would arise from those 
negotiations,'' \citep{world_intellectual_property_organization2013intergovernmental}.
However, such carefully worded statements are part and parcel of diplomatic negotiations, and 
do not necessarily reveal a country's true views.
In order to more fully appreciate the context surrounding the diffusion of traditional knowledge law 
and policy, it is essential to know whether the United States, as the world's foremost 
proponent of stronger intellectual property protection, is truly open to the conclusion of a binding international treaty protecting traditional knowledge.

Interviews with government officials at relevant agencies, such as the United States Trademark and 
Patent Office, are unlikely to obtain information that is any less formulaic than what 
appears in the diplomatic interventions of the U.S. at WIPO. Interviews with employees of WIPO itself 
can be informative, but in the author's experience WIPO officials strive to maintain balance and objectivity, 
and thus are very 
hesitant to discuss details regarding individual member states. Officials from states that fully 
support the negotiation of a binding international treaty on traditional knowledge are willing to be 
frank, but information obtained from such interviews is inherently biased.

Given these challenges, leaked diplomatic cables represent a valuable 
data source, since they contain frank assessments both of the discussions on traditional 
knowledge at WIPO, as well as the policy initiatives on traditional knowledge taking place 
throughout the world. To demonstrate how the leaked cables alter and clarify our  
understanding of the United States' 
latent attitude towards traditional knowledge, consider the following five cables.

\subsubsection{Bogot\'a, Colombia: Traditional knowledge on the radar}
The earliest 
cable, written from the U.S. embassy in Bogot\'a, Colombia, and sent to the State Department and the 
United States Trade Representative (USTR), dates from April 2004, and provides an analysis of intellectual 
property-related matters in Colombia, which were an important issue for a potential U.S.-Andean 
free trade agreement (FTA) being considered at that time.\endnote{Negotiations on a U.S.-Andean FTA 
formally began in May 2004 between the U.S. and Colombia, Ecuador, and Peru. Significant differences 
as to the scope and obligations of the parties arose. Ultimately, only Peru continued the negotiations 
that had begun as the U.S.-Andean FTA, culminating in the U.S.-Peru Trade Promotion Agreement.} 
On the issue of traditional knowledge, the cable 
notes ``Colombia may seek a `sui generis' protection regime in this area. However, developing a 
transparent and objective set of rules on incorporating community participation in patenting 
traditional knowledge would be extremely challenging in 
practice,'' \citep{u.s._department_of_state2004andean}. Even though this 
is a relatively mild comment, it already offers a more frank view of the U.S. position on formal 
intellectual property protection for traditional knowledge than U.S. statements at WIPO.

\subsubsection{Bangkok, Thailand: Politicization of traditional knowledge}
Other cables suggest that the issue of traditional knowledge protection was increasing in political 
importance for foreign countries. A November 2005 cable sent from the U.S. embassy in Bangkok to the 
State Department, 
USTR, and the Department of Agriculture analyzes the state of negotiations surrounding the 
U.S.-Thailand FTA.\endnote{Negotiations on a U.S.-Thailand FTA began in June 2005, but faced 
significant domestic opposition in Thailand. A military coup in 2006 removed the Thai Prime Minister from 
office; since that time, no negotiations have taken place.}
The cable's author reports that the Thai Ministry of Foreign Affairs (MFA) refused to negotiate on  
intellectual property rights unless the U.S. government was willing to discuss traditional knowledge: 
``MFA ostensibly decided against sending a team to discuss IPR at the London round after learning that 
the USG [U.S. government] would only have negotiators on hand to discuss enforcement, i.e., no experts were available to 
discuss Geographical Indications and Traditional 
Knowledge,'' \citep{u.s._department_of_state2005u.s.-thailand}.

\subsubsection{La Paz, Bolivia: Traditional knowledge and industrial property in conflict}
At first, the U.S. government appears to have viewed traditional knowledge as a political annoyance, 
an issue that might be relevant within the context of FTA negotiations, but otherwise was 
unimportant. However, later cables show that the U.S. had come to view demands for the protection 
of traditional knowledge as competing with other forms of intellectual property. For example, 
in late 2007 and amid significant unrest, Bolivian President Evo Morales introduced a new 
constitution, which was later approved in a referendum. Due to the fact that Morales was the 
leader of a democratic socialist political party and had 
long harbored anti-U.S. sentiments, the United States was gravely concerned about the ramifications 
of the new constitution. U.S. embassy staff in La Paz obtained a draft copy of the constitution from a 
member of a minority political party, and sent an analysis back to Washington. The author of the cable 
frames the issue of traditional knowledge as a conflict between a new form of intellectual property 
and existing forms of intellectual property:
\begin{quote}Intellectual property
rights for traditional knowledge and cultural items would be emphasized, with the State being required 
to set up a register for collectively-owned traditional knowledge. (Comment: The Bolivian IP agency 
SENAPI is already focusing on this registry, to the detriment of industrial property claims. 
End comment.) \citep{u.s._department_of_state2007bolivia:}
\end{quote}

\subsubsection{Bras\'ilia, Brazil: Concern about national initiatives to protect traditional knowledge}
In public fora like WIPO, the United States suggests that national initiatives on traditional knowledge 
are preferable to international initiatives, both because national initiatives can account for 
widely varying local conditions, and because national initiatives can offer important information for 
the future development of international initiatives. Privately, however, the United States expresses 
concern about national initiatives to protect traditional knowledge. As an illustrative example, 
consider comments from the U.S. embassy in Bras\'ilia regarding Brazil's approach to regulating access 
to genetic resources and traditional knowledge:
\begin{quote}
Brazil has put in place an elaborate and confusing regime for 
controlling access to genetic resources and traditional knowledge. 
This regime comes at a price to Brazil and the world.  Anecdotal 
evidence from Brazilian and U.S. government and non-governmental 
sources suggests that there has been a marked decline in scientific 
research in Brazil involving genetic resources and traditional 
knowledge \citep{u.s._department_of_state2009brazils}.
\end{quote}

\subsubsection{Jakarta, Indonesia: Avoiding references to traditional knowledge in bilateral agreements}
The growing concern over attempts to protect traditional knowledge resulted 
in the U.S. avoiding explicit references to traditional 
knowledge in 
bilateral agreements.\endnote{While the U.S.-Peru and U.S.-Panama Trade Promotion Agreements both make 
brief reference to traditional knowledge in the context of intellectual property, 
the references occur outside the formal text of the agreement in additional ``understandings'' and 
``side letters.''}
A cable from the U.S. embassy in Jakarta, written in 
late 2009, discusses progress on negotiations surrounding a U.S.-Indonesia Science and Technology 
Agreement within the larger context of U.S-Indonesia relations and U.S. engagement with the Muslim 
world. The cable's author reports that traditional knowledge remained a final area of contention, and 
writes ``We have informed DEPLU [the Indonesian Ministry of Foreign Affairs] and other
stakeholders that the USG [U.S. government] cannot agree to any explicit reference to
GRTKF [genetic resources, traditional knowledge, and 
folklore],'' \citep{u.s._department_of_state2009indonesia}.

% Could discuss TPP proposed chapter on TK here

These five cables represent only a few of the dozens of cables in the Cablegate corpus that discuss 
the topic of 
traditional knowledge.\endnote{While text mining methods can be successfully applied even to very small datasets, such methods generally improve in reliability and return on investment when applied to larger 
datasets \citep{sordo2005sample,kana2012automated}. 
Opting for a traditional qualitative approach in this section also serves to demonstrate that large scale leaks 
offer significant value even for researchers unfamiliar with more technical methods.} Based on my reading of the corpus and other research, they 
provide a relatively representative 
view of the United States' positions. Their value comes from the fact that they present a very different 
picture of U.S. policy 
on and attitudes towards traditional knowledge than could be gleaned from public statements at WIPO or 
from interviews. Furthermore, they were written by different embassy staff 
across a period of 
several years. Relying on a single cable leaves researchers susceptible to an 
individual author's idiosyncrasies, but relying on multiple cables from multiple regions and 
several years largely addresses this concern. As noted above, traditional knowledge is a narrow topic 
within the field of intellectual property, which itself is only a single aspect of economic and 
trade policy. In spite of this relatively narrow focus, the Cablegate corpus contains a significant 
amount of information about traditional knowledge. Researchers who assume that their topic is too 
narrow or policy-focused to be discussed in the Cablegate corpus may be surprised by its breadth of coverage.

\section{Conclusion}

Despite the breadth and depth of information exposed by a series of high-profile leaks over the past several years, 
political scientists have been relatively reluctant to use leaked information as a data source in research. 
In contrast, scholars in other disciplines have already published findings relying upon leaked information. 

Political scientists' reluctance may stem from methodological concerns about data quality or selection 
bias, or from concerns about the ethical, legal, and professional implications of using leaked information. 
It is not the case that the methodological objections to using leaked information 
as a data source are invalid or baseless; rather, there is little reason \emph{a priori} to assume 
that leaked information is any worse in these respects than other commonly used data sources. 
Researchers could benefit from closer attention to data quality and 
selection bias irrespective of their sources.
Likewise, the ethical objections frequently raised in the context of leaked information are 
typically far more relevant to those dealing with non-public information or the media. The legal concerns have some basis, 
particularly for governmental employees or researchers who have or may in the future obtain access to 
classified information; however, much of the Cablegate corpus is likely immune from such concerns, 
and it is very hard to see how academic research could be construed as harming national security.

Alternatively, our reluctance may be related to skepticism about whether leaked information is relevant 
to our research, and if so, whether it can provide any additional value above and beyond other sources. 
Yet far from simply being gossip or confirming what we already know, leaked information provides 
valuable insights into a wide variety of areas that are relevant to research in political science. 
Scholars from other disciplines have so far focused on conflict modeling and prediction, but as I 
have shown, leaks can offer insight into negotiating dynamics and specific areas of economic and 
trade policy, such as intellectual property.

Readers of this journal in particular will be interested in the contents of Cablegate, which contains 
significant amounts of material relating to U.S. diplomatic efforts on global public health, international 
development, labor and environmental standards, climate change negotiations, and many other policy-relevant topics. 
Diplomatic cables also offer insight into what might otherwise be ``dogs that do not bark,'' as embassy officials and 
bureaucrats provide explanations when negotiations fail.
Broader analyses could employ text-mining techniques to analyze shifts in language and tone over 
time \citep{paruchuri2012tracking}.

Leaked information offers an expansive data source for political scientists interested in a wide variety of topics. 
By avoiding engagement with leaked information, we are unnecessarily limiting ourselves, our research, and 
the state of knowledge. The methodological, ethical, and legal objections to the use of such information 
are not insurmountable, and the information itself is uniquely valuable. When adequate alternatives 
do not exist and the value of our research outweighs the potential harm some marginal additional publicity may cause, 
we can and should move forward with using leaked information.

% in the endnotes, we change it without `\textsuperscript`, adding a space
%\patchcmd{\theendnotes}
%  {\makeatletter}
%  {\makeatletter\renewcommand\makeenmark{\theenmark }}
%  {}{}

\section{Appendix}

In order to facilitate replication and further analysis, the data and code used to 
generate the figures, tables, and other statistics in this article are available online 
in a GitHub repository at [redacted for peer review]. This appendix briefly summarizes the methods used in the 
article. Section \ref{tpp_ip} uses a Python script employing regular expressions to extract groups of country 
codes from the draft text, and stores these groups in a flat file. I read these data 
into R, where I generate counts of dyad frequencies and use the \emph{igraph} package to 
create Figure \ref{fig_tpp_network_graph} based on these counts.
In Section \ref{tpp_np}, I manually code flat files corresponding 
to the leaked table recording negotiating positions; because the leaked document was released as a PDF image, 
I had to recreate its data in an accessible format. R provides standard functions to generate distance matrices, 
which I use both to experiment with clustering and as input to R's metric MDS 
command, \emph{cmdscale}. MDS produces the set of coordinates plotted in Figure \ref{fig_tpp_chapters}.

\theendnotes

\bibliographystyle{apa}
\bibliography{leaks}

\end{document}

% Todo:

%%%%%%%% Discuss the limitations of text mining using public diplomatic statements
